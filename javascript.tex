\section{Javascript, d3.js, and d3po.js}

Javascipt offers many ways to create data-driven graphics. A popular solution is \href{http://d3js.org/}{D3.js}, a JavaScript library to create and control web-based dynamic and interactive graphical forms. A gallery of some beautiful d3.js plots can be found \href{https://github.com/mbostock/d3/wiki/Gallery}{here}.

Authorea now supports most Javascript-based data visualization solution. The example below - Figure \ref{fig:1} - is a plot generated using \href{http://d3pojs.org/}{D3po.js} which is an extension of d3.js also written in Javascript. D3po allows anyone with no special data visualization skills, to make an interactive, publication-quality figure that has staged builds and linked brushing through scatter plots. What's even cooler is that the plot below is based on actual data (astrophysics data, yay!). The figure describes how metallicity affects color in cool stars. It is based on work of graduate student Elizabeth Newton and others \cite{2014AJ....147...20N}. \textbf{Try clicking and dragging in the scatter plots} to understand the power of linked brushing in published figures.

You should know that \textbf{this entire visualization is running within Authorea}. The Javascript, HTML, CSS and \textbf{all the data} associated with this image are all part of this blog post. They are individual files which can be found by clicking on the folder icon on the top left corner of this page.

 