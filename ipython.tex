\section{IPython Notebooks}
Another very popular solution among scientists for bundling together research data, analysis, and code is the \href{http://ipython.org/notebook.html}{IPython Notebook}. IPython NB is a web-based interactive computational environment where you can combine code execution, text, mathematics, \href{http://jakevdp.github.io/blog/2013/12/05/static-interactive-widgets/}{plots and rich media} into individual documents. These documents can now be imported, visualized and executed within Authorea! herh er hreh r 

So, if you already use IPython Notebook to collect all your analysis and generate your plots, you can now include these notebooks in your published articles on Authorea. Below, in Figure \ref{fig:2}, we show a plot generated by Python code which uses the \textbf{SymPy, Numpy and Matplotlib packages} to perform symbolic manipulations. It is a numerical visualization of symbolically constructed expressions.

If you \textbf{hover on the figure below}, you will notice a link in the lower left corner: \textbf{Launch IPython}. By clicking it, you will launch an environment where you can visualize and interact with this plot (follow the instructions). For example, you can \textbf{change the parameters} that were used to generate it and see it change interactively. You can also \textbf{download} the entire Notebook and play it with it locally.  